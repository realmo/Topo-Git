\documentclass[a4paper]{article}
% Options possibles : 10pt, 11pt, 12pt (taille de la fonte)
%                     oneside, twoside (recto simple, recto-verso)
%                     draft, final (stade de développement)

\usepackage[utf8]{inputenc}   % LaTeX, comprends les accents !
\usepackage[T1]{fontenc}      % Police contenant les caractères français
\usepackage[francais]{babel}  % Placez ici une liste de langues, la
															% dernière étant la langue principale
\usepackage{float}
\usepackage[xindy]{glossaries}
\makeglossaries

\usepackage{graphicx}					% pour inclure des graphiques
\usepackage[a4paper]{geometry}% Réduire les marges
%\pagestyle{headings}        	% Pour mettre des entêtes avec les titres
                              % des sections en haut de page
 


\title{Topo Git}           		% Les paramètres du titre : titre, auteur, date
%\author{Jean-Loup H.}					% on peut ajouter \and Co-auteur
\date{}                       % La date n'est pas requise (la date du
                              % jour de compilation est utilisée si rien d'autres n'est spécifié)

%\sloppy                       % Ne pas faire déborder les lignes dans la marge
\newacronym[first={VCS}]{VCS}{VCS}{Version Control System}
\newacronym[first={RCS}]{RCS}{RCS}{Revision Control System}
\begin{document}

\maketitle                    % Faire un titre utilisant les données
                              % passées à  \title, \author et \date
\begin{center}
	\includegraphics[width=145px,height=57px]{illustrations/6-30-12_Git.jpg}
\end{center}
							  % Si on veut un résumé on utilise :
\begin{abstract}
Git est présent aujourd'hui dans un grand nombre de projets et apparait comme un choix de fait dans une large majorité des projets émergents. Cela s'explique par les innovations qu'il apporte dans le contrôle de version, notamment en terme de souplesse pour le développeur.
Toutefois, appréhender ce système de contrôle de version \gls{VCS} sous-entend une modification de 

\end{abstract}
\tableofcontents			  % si on veut une table des matière

% \part{Titre}                % Commencer une partie...

\section{Particularités}      % Commencer une section, etc.

Après \gls{RCS} et jusqu'à Git, les systèmes de contrôle de version sont restés basés sur la même idée. C'est à dire, un principe reposant sur une copie des fichiers.
%\subsection{Praesent}        % Section plus petite

% \subsubsection{Titre}       % Encore plus petite

%\subsection{Quisque}


% \paragraph{Titre}           % Toutes petites sections (le nom \paragraph
                              % n'est pas très bien choisi)

% \subparagraph{Titre}        % La dernière

% \appendix                   % Commençons les annexes

% \section{Titre}             % Annexe A

% \section{Titre}             % Annexe B

% \listoffigures              % Table des figures

% \listoftables               % Liste des tableaux
\section{Glossaire}
\printglossaries

\end{document}
